\documentclass[12pt,a4paper]{extarticle}

\usepackage{cmap}                   
\usepackage{mathtext}               
\usepackage[T1,T2A]{fontenc}        
\usepackage[utf8]{inputenc}         
\usepackage[english, russian]{babel} 

\usepackage[top=0.35in, bottom=0.5in, left=0.3in, right=0.3in]{geometry}
\usepackage{mathtools}              
\mathtoolsset{showmanualtags,mathic,centercolon}
\usepackage{amssymb}                
\usepackage{amsthm}                 
\usepackage{amstext}                
\usepackage{amsfonts}               
\usepackage{icomma}                 
\usepackage{enumitem}              
\usepackage{array}                  
\usepackage{multirow}
\usepackage{setspace}

\usepackage{algorithm}              
\usepackage{algorithmicx}           
\usepackage[noend]{algpseudocode}   
\usepackage{listings}              
\renewcommand{\algorithmicrequire}{\textbf{Input:}}              
\renewcommand{\algorithmicensure}{\textbf{Output:}}              
\floatname{algorithm}{Algorithm}                                 
\renewcommand{\algorithmiccomment}[1]{\hspace*{\fill}\{// #1\}}
\newcommand{\algname}[1]{\textsc{#1}}                          
\usepackage{physics}

\usepackage{euscript}               
\usepackage{mathrsfs}               

%% Графика
\usepackage{graphicx}       
\graphicspath{{images/}}            
\usepackage{tikz}  
\usetikzlibrary{patterns}                 
\usepackage{pgfplots}              
\usepackage{circuitikz}


\usepackage{indentfirst}                    
\usepackage{epigraph}                       
\usepackage{fancybox,fancyhdr}              
\usepackage[colorlinks=true,citecolor=blue]{hyperref} 
\usepackage{titlesec}                       
\usepackage[normalem]{ulem}                 
\usepackage[makeroom]{cancel}               
\usepackage{dsfont}

\usepackage{diagbox}
\usepackage{makecell}

\usepackage{csquotes}

\mathtoolsset{showonlyrefs=true}        
\renewcommand{\headrulewidth}{1.8pt}    
\renewcommand{\footrulewidth}{0.0pt}    

\usepackage{forest} 

\usetikzlibrary{arrows,calc}
\usetikzlibrary{quotes,angles}

\usetikzlibrary{positioning,intersections}

\usetikzlibrary{through}

\usepackage{enumitem}

\newenvironment{turing}[2]
{\begin{enumerate}[leftmargin=0pt,labelsep=0pt,align=left,parsep=0pt]
		\item[$#1={}$]``\ignorespaces#2
%		\begin{enumerate}[
			nosep,
			align=left,
			labelwidth=1.5em,
			label=\bfseries\arabic{*}.,
			ref=\arabic{*}
			]}
		{\unskip''\end{enumerate}\end{enumerate}}

\newcommand{\bitem}{\item\hspace*{1em}\ignorespaces}

\usepackage{graphicx}

\newtheorem{definition}{Definition}[section]

\newtheorem*{task}{Task}
\newtheorem*{task0}{Task 0}
\newtheorem*{task1}{Task 1}
\newtheorem*{task2}{Task 2}
\newtheorem*{task3}{Task 3}
\newtheorem*{task4}{Task 4}
\newtheorem*{task5}{Task 5}
\newtheorem*{task6}{Task 6}
\newtheorem*{task7}{Task 7}
\newtheorem*{task8}{Task 8}
\newtheorem*{task9}{Task 9}
\newtheorem*{task10}{Task 10}
\newtheorem*{task11}{Task 11}
\newtheorem*{task12}{Task 12}

\newtheorem{theorem}{Theorem}
\newtheorem{proposal}{Proposal}
\newtheorem{notice}{Notice}
\newtheorem{statement}{Statement}
\newtheorem{corollary}{Corollary}
\newtheorem{lemma}{Lemma}
\newtheorem{observation}{Observation}
\newtheorem{problem}{Problem}
\newtheorem{claim}{Claim}


\newcommand{\note}{\underline{Note:} }
\newcommand{\fact}{\underline{\textbf{Fact}:} }
\newcommand{\example}{\underline{Example:} }


\renewcommand{\Re}{\mathrm{Re\:}}
\renewcommand{\Im}{\mathrm{Im\:}}
\newcommand{\Arg}{\mathrm{Arg\:}}
\renewcommand{\arg}{\mathrm{arg\:}}
\newcommand{\Mat}{\mathrm{Mat}}
\newcommand{\id}{\mathrm{id}}
\newcommand{\aut}{\mathrm{aut}}
\newcommand{\isom}{\xrightarrow{\sim}} 
\newcommand{\leftisom}{\xleftarrow{\sim}}
\newcommand{\Hom}{\mathrm{Hom}}
\newcommand{\Ker}{\mathrm{Ker}\:}
\newcommand{\rk}{\mathrm{rk}\:}
\newcommand{\diag}{\mathrm{diag}}
\newcommand{\ort}{\mathrm{ort}}
\newcommand{\pr}{\mathrm{pr}}
\newcommand{\vol}{\mathrm{vol\:}}
\renewcommand{\mod}{\mathrm{\: mod\:}}
\DeclareMathOperator*\lowlim{\underline{lim}}
\DeclareMathOperator*\uplim{\overline{lim}}
\newcommand{\nd}{\mathbin{\&}}

\newcommand{\X}{\mathbb{X}}
%\newcommand{\D}{\mathbb{D}}
\newcommand{\Y}{\mathbb{Y}}
%\newcommand{\I}{\mathbb{I}}
\makeatletter
\DeclareRobustCommand{\I}{\operatorname{\mathds{I}}\@ifstar\@firstofone\@I}
\newcommand{\@I}[1]{\left\{#1\right\}}
\makeatother

\newcommand{\Z}{\mathbb{Z}}
\newcommand{\Qq}{\mathcal{Q}}
\newcommand{\N}{\mathbb{N}}
%\newcommand{\E}{\mathbb{E}} %
\makeatletter
\DeclareRobustCommand{\E}{\operatorname{\mathds{E}}\@ifstar\@firstofone\@E}
\newcommand{\@E}[1]{\left[#1\right]}
\makeatother

\makeatletter
\DeclareRobustCommand{\D}{\operatorname{\mathbb{D}}\@ifstar\@firstofone\@D}
\newcommand{\@D}[1]{\left[#1\right]}
\makeatother

\makeatletter
\DeclareRobustCommand{\Pr}{\operatorname{\mathds{P}}\@ifstar\@firstofone\@Pr}
\newcommand{\@Pr}[1]{\left[#1\right]}
\makeatother

\makeatletter
\DeclareRobustCommand{\cov}{\operatorname{\mathrm{cov}}\@ifstar\@firstofone\@cov}
\newcommand{\@cov}[1]{\left(#1\right)}
\makeatother

\renewcommand{\S}{\mathbb{S}}
\newcommand{\Q}{\mathbb{Q}}
\newcommand{\R}{\mathbb{R}} 
\newcommand{\B}{\mathbb{B}}
\renewcommand{\C}{\mathbb{C}}
\renewcommand{\L}{\mathscr{L}}
%\renewcommand{\P}{\mathds{P}}


\newcommand{\orthog}{\mathop{\bot}}
\renewcommand*\d{\mathop{}\!\mathrm{d}}
\renewcommand*\dd{\mathop{}\!\partial}

%\renewcommand{\Pr}{\mathds{P}}
\newcommand{\pn}{\xrightarrow{\text{a. s.}}}
\newcommand{\pp}{\xrightarrow{\mathds{P}}}
\newcommand{\pd}{\xrightarrow{d}}
\newcommand{\ra}{\rightarrow}


\newcommand{\fe}{\varphi}
\newcommand{\e}{\varepsilon}
\newcommand{\ind}{\mathbin{\perp\!\!\!\perp}}
\newcommand{\Gauss}{\mathrm{Gauss}}
\newcommand{\hence}{\longrightarrow}
\newcommand{\bto}{\Longrightarrow}
\newcommand{\Bin}{\mathrm{Bin}}
\newcommand{\Bern}{\mathrm{Bern}}
\newcommand{\Geom}{\mathrm{Geom}}
\newcommand{\Uni}{\mathrm{U}}
\newcommand{\Exp}{\mathrm{Exp}}
\newcommand{\Ko}{\mathrm{Ko}}
\newcommand{\No}{\mathcal{N}}
\newcommand{\Pois}{\mathrm{Pois}}
\newcommand{\filtr}{\mathcal{F}}
\newcommand{\Filtr}{\mathbb{F}}

     
\title{\Huge{ДЗ №5, Винеровский процесс}}
\author{Павел Захаров}
\date{}
     
     
\begin{document}
	\maketitle

	
	\vspace{\baselineskip}

	
	
	\vspace{\baselineskip}
	\begin{task1}
		Пусть $(W_t, t \geq 0)$ -- винеровский процесс. Докажите, что следующие процессы тоже винеровские
		\begin{itemize}
			\item[а)] $X_t = \sqrt{c} W_{t/c}, c > 0$,
			
			\item[б)] $X_t = W_{t+a} - W_a, a > 0$,
			
			\item[в)] $X_t = W_t \I {t < T} + (2W_T - W_t) \I {t \geq T}$
			
			\item[г)]
			\[
				X_t = 
				\begin{cases}
					tW_{1/t}, & t > 0;
					\\
					0, & t = 0.
				\end{cases}
			\]
			
		\end{itemize}
	\end{task1}
	\begin{proof}[Решение]
		\
		\begin{itemize}
			\item[а)] Проверим явно по определению:
			\begin{itemize}
				\item[$\bullet$]		
				$
					X_0 = \sqrt{c} W_{0} \overset{a.s.}{=} \sqrt{c} \cdot 0 = 0
				$
				
				\item[$\bullet$] Рассмотрим вектор $(X_{t_0}, X_{t_1} - X_{t_0}, \ldots ,X_{t_n} - X_{t_{n - 1}}) = \sqrt{c}(W_{t_0/c}, W_{t_1/c} - W_{t_0/c}, \ldots , W_{t_n/c} - W_{t_{n - 1}/c})$, компоненты 
				этого вектора независимы, так как $(W_t, t \geq 0)$ есть процесс с независимыми приращениеми.
				
				\item[$\bullet$]
				\[
					X_t - X_s = \sqrt{c} \left( W_{t / c} - W_{s / c} \right)
					\sim
					\sqrt{c} \No(0, (t-s)/c) \sim \No(0, t - s).
				\]
				Все три пункта определения выполняются, следовательно процесс Винеровский.
			\end{itemize}

			
			\item[б)] 
			Воспользуемся эквивалентным определением винеровского процесса для процесса $(W_t, t > 0)$ и покажем, что все условия выполнены и для процесса $X_t$.
			\begin{itemize}
				\item[$\bullet$]
				Рассмотрим вектор $(X_{t_1}, X_{t_2}, \ldots, X_{t_{n-1}}, X_{t_n})$, где $0 \leq t_1 < t_2 < \ldots < t_n$. 
					
				Он был получен линейным преобазованием вектора $(X_{t_1},  X_{t_2} - X_{t_1}, \ldots, X_{t_n} - X_{t_{n - 1}})
				=$ 
				
				$=(W_{t_1 + a}, W_{t_2 + a} - W_{t_1 + a}, \ldots, W_{t_n + a} - W_{t_{n - 1} + a})$, а последний в свою очередь является Гауссовским, так как все компоненты независимы и нормально распределены.
				\item[$\bullet$] $\E {X_t} = \E {W_{t + a}} - \E {W_{a}} = 0 - 0 = 0$.
				
				\item[$\bullet$]
				\[
					\cov{X_s, X_t} = \cov{W_{s+a} - W_{a}, W_{t + a} - W_a}
					=
				\]
				\[
					=
					\cov{W_{s + a}, W_{t + a}} - \cov{W_{a}, W_{t + a}} - \cov{W_{s + a}, W_{a}} +\cov{ W_{a}, W_{a} }
					 = 
				\]
				\[
					= \min(s + a, t + a)  - a - a + a = \min(s, t).
				\]
				Все три условия выполнены, следовательно процесс винеровский.
			\end{itemize}
			

			
			\item[в)] 
			Применим эквивалентное определение:
			\begin{itemize}
				\item[$\bullet$] Рассмотрим $(X_{t_1}, \ldots, X_{t_n})$. Он состоит из линейной комбинации вектора $(W_{t_1}, \ldots, W_{t_n}, W_T)$, который является гауссовским.
				
				\item[$\bullet$] 
				Если $t < T$, то $\E{X_t} = \E {W_t} = 0$
				
				Иначе $\E {X_t} = 2\E {W_T} - \E {W_t} = 2 \cdot 0 - 0 = 0$.
				
				\item[$\bullet$]
				\[
					\cov{X_s, X_t} = \cov{W_s \I {s < T} + (2W_T - W_s) \I {s \geq T}, W_t \I {t < T} + (2W_T - W_t) \I {t \geq T}}
				\]
				Рассмотрим случаи:
				\begin{itemize}
					\item $\max(t, s) < T$:
					\[
						\cov{X_s, X_t} = \cov{W_s, W_t} = \min(s, t)
					\]
					\item $\min(t, s) \geq T$:
					\[
						\cov{X_s, X_t} = \cov{2W_T - W_s, 2W_T - W_t} = 
					\]
					\[
						=
						4\D {W_T} - 2 \cov{W_s, W_T} - 2\cov {W_t, W_T} + \cov{W_t, W_s} = 
					\]
					\[
						=
						4T - 2T - 2T + \cov{W_t, W_s} = \min(t, s).
					\]
					\item $\min(t, s) < T, \max(t, s) \geq T$:
					\[
						\cov{X_s, X_t} = \cov{W_{\min(s, t)}, 2W_T - W_{\max(s, t)}}
						=
					\]
					\[
						=2\cov{W_{\min(s, t)}, W_T} - \cov{W_{\min(s, t)}, W_{\max(s, t)}}
						=
						2\min(s, t) - \min(s, t) = \min(s, t).
					\]
			\end{itemize}
			Во всех случаях третье условие выполняется.		
			
			\end{itemize}
			Получается, процесс винеровский по эквивалентному определению.
			
			\item[г)]
			Проверим все три условия из эквивалентного определения:
			\begin{itemize}
				\item[$\bullet$] $(X_{t_1}, X_{t_2}, \ldots, X_{t_{n-1}}, X_{t_n}) = (t_1\cdot W_{t_1}, t_2\cdot W_{1/t_2}, \ldots, t_{n-1}\cdot X_{1/t_{n-1}}, t_n\cdot W_{1/t_n})$. 
				
				Но вектор $(W_{t_1}, W_{1/t_2}, \ldots, W_{1/t_{n-1}}, W_{1/t_n})$ гауссовский. Но вышенаписанный вектор есть этот вектор, умноженный на $A = \diag(t_1, \ldots, t_n)$. Тогда он тоже гауссовский.
				
				\item[$\bullet$] $\E {X_t} = t \E {W_{1/t}} = t \cdot 0 = 0$.
				
				\item[$\bullet$]
				\[
					\cov{X_t, X_s} = \cov{t W_{1/t}, s W_{1/s}}
					=
					ts\cov{W_{1/s}, W_{1/t}}  = ts\min (1/t, 1/s) = {ts \over \max(t, s)} = \min(t,s).
				\]
				Получается, что процесс действительно винеровский.
 			\end{itemize}
		\end{itemize}
	\end{proof}
	
	
	
	
	
	
	
	
	
	
	
	
	
	
	
	\newpage
	

	\begin{task2}
		Докажите локальный закон повторного логарифма для винеровского процесса:
		\[
			\Pr{\limsup _{t \rightarrow 0+} {W_{t} \over \sqrt{2 t \ln \ln (1 / t)}} = 1} = 1.
		\]
	\end{task2}
	\begin{proof}[Решение]
		\
		Определим новый случайный процесс: 
		\[
			X_t = 
			\begin{cases}
			tW_{1/t}, & t > 0;
			\\
			0, & t = 0.
			\end{cases}
		\]
		По пункту г) задачи 1 знаем, что этот процесс является Винеровским.
		Посмотрим на ЗПЛ для этого процесса:
		\[
			1 = \Pr{\limsup _{t \rightarrow +\infty} {X_{t} \over \sqrt{2 t \ln \ln t}} = 1}
			=
			\Pr{\limsup _{t \rightarrow +\infty} {W_{1 \over t} \over \sqrt{2 t^{-1} \ln \ln t}} = 1}
			=
		\]
		\[
			=\left\{r = {1 \over t}\right\}
			=
			\Pr{\limsup _{r \rightarrow 0+} {W_r \over \sqrt{2 r \ln \ln (1/r)}} = 1}.
		\]
		Что и требовалось доказать.
	\end{proof}
	
	
	\newpage
	
	
	
	
	
	
	
	
	
	
	
	
	\begin{task3}
		Докажите, что существует гауссовский процесс $X = (X_t , t \in \R^d_+)$ с нулевой функцией среднего и ковариационной функцией
		\[
			R(s, t) = \prod_{k=1}^d \min(s_k, t_k),
		\]
		где $s = (s_1 , \ldots , s_d) \in \R^d_+$, $t = (t_1 ,\ldots , t_d ) \in \R^d_+$.
	\end{task3}
	\begin{proof} [Решение]
		\
		Знаем, что если ковариационная функция симметричная и неотрицательно определенная, то существует гауссовский процесс с нулевым средним и такой ковариационной функцией.
		
		Про симметричность понятно. Проверим $R(s, t)$ на неотрицательную определенность:
		\[
			\sum_{i, j = 1}^n x_i x_j\prod_{k=1}^d \min(t_{i, k}, t_{j, k}) \overset{?}{\geq} 0, \quad x_i \in \R
		\]
		\[
			\sum_{i, j = 1}^n x_i x_j\prod_{k=1}^d \min(t_{i, k}, t_{j, k})
			=
			\sum_{i, j = 1}^n x_i x_j\prod_{k=1}^d \int\limits_{0}^{+\infty} \I {z \leq t_{i,k}} \I {z \leq t_{j,k}} \d z
			=
		\]
		\[
			=
			\sum_{i, j = 1}^n x_i x_j \int\limits_{0}^{+\infty}\ldots \int\limits_{0}^{+\infty} \prod_{k=1}^d \left(\I{z_k \leq t_{i,k}} \I {z_k \leq t_{j,k}}\right) \d z_1 \ldots \d z_d
			=
		\]
		\[
			=
			\int\limits_{0}^{+\infty}\ldots \int\limits_{0}^{+\infty} \sum_{i, j = 1}^n x_i x_j\prod_{k=1}^d \left(\I{z_k \leq t_{i,k}} \I {z_k \leq t_{j,k}}\right) \d z_1 \ldots \d z_d
			=
		\]
		\[
			=
			\int\limits_{0}^{+\infty}\ldots \int\limits_{0}^{+\infty} \sum_{i, j = 1}^n \left(x_i \prod_{k=1}^d \I{z_k \leq t_{i,k}}\right)\left(x_j \prod_{k=1}^d \I{z_k \leq t_{j,k}}\right)  \d z_1 \ldots \d z_d
			=
		\]
		\[
			=
			\int\limits_{0}^{+\infty}\ldots \int\limits_{0}^{+\infty} \sum_{i, j = 1}^n \left(x_i \prod_{k=1}^d \I{z_k \leq t_{i,k}}\right)^2 \d z_1 \ldots \d z_d \geq 0.
		\]
		
	\end{proof}
	
	
	
	
	
	
	
	
	
	
	
	
	
	\newpage
	
	
	\begin{task4}
		Гауссовский процесс $(X_t , t \geq 0)$ имеет нулевую функцию среднего и ковариационную функцию $R(s, t) = e^{-\max(t, s)} - e^{-s - t}$. Докажите, что такой процесс	существует и что процесс:
		\[
			Y_t = (t+1)X_{\ln{t+1 \over t}} \I {t > 0}
		\] является винеровским.
	\end{task4}
	\begin{proof} [Решение]
		Знаем, что если ковариационная функция симметричная и неотрицательно определенная, то существует гауссовский процесс с нулевым средним и такой ковариационной функцией.
		
		Про симметричность понятно. Проверим $R(s, t)$ на неотрицательную определенность:
		\[
			\sum_{i, j = 1}^n x_i x_j\left(e^{-\max(t_i, t_j)} - e^{-t_i - t_j}\right) \overset{?}{\geq} 0, \quad x_i \in \R
		\]
		Так как мы проходимся по всем парам $t_k$, то можем перемешать их так, чтобы $t_i$ было больше чем $t_{i+1}$. Тогда перепишем сумму:
		\[
			\sum_{i, j = 1}^n x_i x_j\left(e^{-\max(t_i, t_j)} - e^{-t_i - t_j}\right)
			=
			2\sum_{i < j}^n x_i x_j\left(e^{-\max(t_i, t_j)} - e^{-t_i - t_j}\right)
			+\sum_{i = 1}^n x_i^2\left(e^{-\max(t_i, t_i)} - e^{-t_i - t_i}\right)
			=
		\]
		\[
			=
			2\sum_{i < j}^n x_i x_j\left(e^{-t_i} - e^{t_i + t_j}\right)
			+\sum_{i = 1}^n x_i^2\left(e^{-t_i} - e^{-2t_i}\right)
		\]
		Заметим, что на выходе у нас получается то, что в простонародье называется квадратичной формой. Выпишем её матрицу:
		\[
			M = 
			\begin{pmatrix}
				e^{-t_1} - e^{-2t_1} & 	e^{-t_1} - e^{-t_1 - t_2} & \dots & e^{-t_1} - e^{-t_1 - t_n}
				\\
				e^{-t_1} - e^{-t_1 - t_2} & e^{-t_2} - e^{-2t_2} & \dots & e^{-t_2} - e^{-t_2 - t_n}
				\\
				\vdots &  \vdots & \ddots & \vdots
				\\
				e^{-t_1} - e^{-t_1 - t_n} & e^{-t_2} - e^{-t_2 - t_n} & \dots & e^{-t_n} - e^{-2t_n}
			\end{pmatrix}
		\]
		
		\vspace{\baselineskip}
		Здесь не вышло. Давайте попробуем вторую часть:
		Проверим все три условия из эквивалентного определения в.п. для $Y_t$:
		\begin{itemize}
			\item[1)] $\E {Y_t} = (t + 1) \cdot 0 = 0$
			
			\item[2)] Вектор с.в. из процесса $Y_t$ получается из вектора с.в. процесса $X_t$ путем домножения на вектор чисел $\{t_i + 1\}$. Тогда он гауссовский.
			
			\item[3)] 
			\[
				\cov{Y_t, Y_s}
				=
				\cov{(t+1)X_{\ln{t+1 \over t}}, (s+1)X_{\ln{s+1 \over s}}}
				=
			\]
			\[
				=
				(t+1)(s+1)\cov{X_{\ln{t+1 \over t}}, X_{\ln{s+1 \over s}}}
				=
				(t+1)(s+1)\left(e^{-\max(\ln{t+1 \over t}, \ln{s+1 \over s})} - e^{-\ln{s+1 \over s} - \ln{t+1 \over t}}\right)
				=
			\]
			\[
				= \{\text{let } t > s\}
				=
				(t+1)(s+1)\left({s \over s + 1} - {st \over (x + 1)(t + 1)}\right) 
				=
				(t+1)s - st = s = \min(t, s).
			\]
			
			Да, все свойства выполняются. Поэтому процесс $Y_t$ винеровский.
			
		\end{itemize}
		
	\end{proof}
	


	
	
	
	
	
	
	
	
	\newpage
	
	
	\begin{task5}
		Докажите, что винеровский процесс $(W_t , t \geq 0)$ с вероятностью 1 является функцией неограниченной вариации на любом конечном отрезке $[a, b] \subset \R_+$.
	\end{task5}
	\begin{proof} [Решение]
		
		
		
	\end{proof}
	

\end{document}