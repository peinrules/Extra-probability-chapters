\documentclass[12pt,a4paper]{extarticle}

\usepackage{cmap}                   
\usepackage{mathtext}               
\usepackage[T1,T2A]{fontenc}        
\usepackage[utf8]{inputenc}         
\usepackage[english, russian]{babel} 

\usepackage[top=0.35in, bottom=0.5in, left=0.3in, right=0.3in]{geometry}
\usepackage{mathtools}              
\mathtoolsset{showmanualtags,mathic,centercolon}
\usepackage{amssymb}                
\usepackage{amsthm}                 
\usepackage{amstext}                
\usepackage{amsfonts}               
\usepackage{icomma}                 
\usepackage{enumitem}              
\usepackage{array}                  
\usepackage{multirow}
\usepackage{setspace}

\usepackage{algorithm}              
\usepackage{algorithmicx}           
\usepackage[noend]{algpseudocode}   
\usepackage{listings}              
\renewcommand{\algorithmicrequire}{\textbf{Input:}}              
\renewcommand{\algorithmicensure}{\textbf{Output:}}              
\floatname{algorithm}{Algorithm}                                 
\renewcommand{\algorithmiccomment}[1]{\hspace*{\fill}\{// #1\}}
\newcommand{\algname}[1]{\textsc{#1}}                          
\usepackage{physics}

\usepackage{euscript}               
\usepackage{mathrsfs}               

%% Графика
\usepackage{graphicx}       
\graphicspath{{images/}}            
\usepackage{tikz}  
\usetikzlibrary{patterns}                 
\usepackage{pgfplots}              
\usepackage{circuitikz}


\usepackage{indentfirst}                    
\usepackage{epigraph}                       
\usepackage{fancybox,fancyhdr}              
\usepackage[colorlinks=true,citecolor=blue]{hyperref} 
\usepackage{titlesec}                       
\usepackage[normalem]{ulem}                 
\usepackage[makeroom]{cancel}               
\usepackage{dsfont}

\usepackage{diagbox}
\usepackage{makecell}

\usepackage{csquotes}

\mathtoolsset{showonlyrefs=true}        
\renewcommand{\headrulewidth}{1.8pt}    
\renewcommand{\footrulewidth}{0.0pt}    

\usepackage{forest} 

\usetikzlibrary{arrows,calc}
\usetikzlibrary{quotes,angles}

\usetikzlibrary{positioning,intersections}

\usetikzlibrary{through}

\usepackage{enumitem}

\newenvironment{turing}[2]
{\begin{enumerate}[leftmargin=0pt,labelsep=0pt,align=left,parsep=0pt]
		\item[$#1={}$]``\ignorespaces#2
%		\begin{enumerate}[
			nosep,
			align=left,
			labelwidth=1.5em,
			label=\bfseries\arabic{*}.,
			ref=\arabic{*}
			]}
		{\unskip''\end{enumerate}\end{enumerate}}

\newcommand{\bitem}{\item\hspace*{1em}\ignorespaces}

\usepackage{graphicx}

\newtheorem{definition}{Definition}[section]

\newtheorem*{task}{Task}
\newtheorem*{task0}{Task 0}
\newtheorem*{task1}{Task 1}
\newtheorem*{task2}{Task 2}
\newtheorem*{task3}{Task 3}
\newtheorem*{task4}{Task 4}
\newtheorem*{task5}{Task 5}
\newtheorem*{task6}{Task 6}
\newtheorem*{task7}{Task 7}
\newtheorem*{task8}{Task 8}
\newtheorem*{task9}{Task 9}
\newtheorem*{task10}{Task 10}
\newtheorem*{task11}{Task 11}
\newtheorem*{task12}{Task 12}

\newtheorem{theorem}{Theorem}
\newtheorem{proposal}{Proposal}
\newtheorem{notice}{Notice}
\newtheorem{statement}{Statement}
\newtheorem{corollary}{Corollary}
\newtheorem{lemma}{Lemma}
\newtheorem{observation}{Observation}
\newtheorem{problem}{Problem}
\newtheorem{claim}{Claim}


\newcommand{\note}{\underline{Note:} }
\newcommand{\fact}{\underline{\textbf{Fact}:} }
\newcommand{\example}{\underline{Example:} }


\renewcommand{\Re}{\mathrm{Re\:}}
\renewcommand{\Im}{\mathrm{Im\:}}
\newcommand{\Arg}{\mathrm{Arg\:}}
\renewcommand{\arg}{\mathrm{arg\:}}
\newcommand{\Mat}{\mathrm{Mat}}
\newcommand{\id}{\mathrm{id}}
\newcommand{\aut}{\mathrm{aut}}
\newcommand{\isom}{\xrightarrow{\sim}} 
\newcommand{\leftisom}{\xleftarrow{\sim}}
\newcommand{\Hom}{\mathrm{Hom}}
\newcommand{\Ker}{\mathrm{Ker}\:}
\newcommand{\rk}{\mathrm{rk}\:}
\newcommand{\diag}{\mathrm{diag}}
\newcommand{\ort}{\mathrm{ort}}
\newcommand{\pr}{\mathrm{pr}}
\newcommand{\vol}{\mathrm{vol\:}}
\renewcommand{\mod}{\mathrm{\: mod\:}}
\DeclareMathOperator*\lowlim{\underline{lim}}
\DeclareMathOperator*\uplim{\overline{lim}}
\newcommand{\nd}{\mathbin{\&}}

\newcommand{\X}{\mathbb{X}}
%\newcommand{\D}{\mathbb{D}}
\newcommand{\Y}{\mathbb{Y}}
%\newcommand{\I}{\mathbb{I}}
\makeatletter
\DeclareRobustCommand{\I}{\operatorname{\mathds{I}}\@ifstar\@firstofone\@I}
\newcommand{\@I}[1]{\left\{#1\right\}}
\makeatother

\newcommand{\Z}{\mathbb{Z}}
\newcommand{\Qq}{\mathcal{Q}}
\newcommand{\N}{\mathbb{N}}
%\newcommand{\E}{\mathbb{E}} %
\makeatletter
\DeclareRobustCommand{\E}{\operatorname{\mathds{E}}\@ifstar\@firstofone\@E}
\newcommand{\@E}[1]{\left[#1\right]}
\makeatother

\makeatletter
\DeclareRobustCommand{\D}{\operatorname{\mathbb{D}}\@ifstar\@firstofone\@D}
\newcommand{\@D}[1]{\left[#1\right]}
\makeatother

\makeatletter
\DeclareRobustCommand{\Pr}{\operatorname{\mathds{P}}\@ifstar\@firstofone\@Pr}
\newcommand{\@Pr}[1]{\left[#1\right]}
\makeatother

\renewcommand{\S}{\mathbb{S}}
\newcommand{\Q}{\mathbb{Q}}
\newcommand{\R}{\mathbb{R}} 
\newcommand{\B}{\mathbb{B}}
\renewcommand{\C}{\mathbb{C}}
\renewcommand{\L}{\mathscr{L}}
%\renewcommand{\P}{\mathds{P}}


\newcommand{\orthog}{\mathop{\bot}}
\renewcommand*\d{\mathop{}\!\mathrm{d}}
\renewcommand*\dd{\mathop{}\!\partial}

%\renewcommand{\Pr}{\mathds{P}}
\newcommand{\pn}{\xrightarrow{\text{a. s.}}}
\newcommand{\pp}{\xrightarrow{\mathds{P}}}
\newcommand{\pd}{\xrightarrow{d}}
\newcommand{\ra}{\rightarrow}


\newcommand{\fe}{\varphi}
\newcommand{\e}{\varepsilon}
\newcommand{\ind}{\mathbin{\perp\!\!\!\perp}}
\newcommand{\Gauss}{\mathrm{Gauss}}
\newcommand{\hence}{\longrightarrow}
\newcommand{\bto}{\Longrightarrow}
\newcommand{\Bin}{\mathrm{Bin}}
\newcommand{\Bern}{\mathrm{Bern}}
\newcommand{\Geom}{\mathrm{Geom}}
\newcommand{\Uni}{\mathrm{U}}
\newcommand{\Exp}{\mathrm{Exp}}
\newcommand{\Ko}{\mathrm{Ko}}
\newcommand{\No}{\mathcal{N}}
\newcommand{\Pois}{\mathrm{Pois}}
\newcommand{\filtr}{\mathcal{F}}
\newcommand{\Filtr}{\mathbb{F}}
     
\title{\Huge{ДЗ №4, Мартингалы и пуассоновский процесс}}
\author{Павел Захаров}
\date{}
     
     
\begin{document}
	\maketitle

	
	\vspace{\baselineskip}

	
	
	\vspace{\baselineskip}
	\begin{task1}
		Пусть задана фильтрация $\Filtr = \left(\filtr_{n}, n \in \mathbb{N}\right)$, а $\tau_{1}, \tau_{2}, \ldots$ -- марковские моменты
		относительно $\Filtr$. Докажите, что случайные величины
		\[
			\sum_{k=1}^{m} \tau_{k}, \prod_{k=1}^{m} \tau_{k}, \quad \sup _{k} \tau_{k}, \quad \inf _{k} \tau_{k}
		\]
		тоже являются марковскими моментами относительно $\Filtr$.
	\end{task1}
	\begin{proof}[Решение]
		\
		Так как $\forall i \in \N : \tau_i$ -- Марковский процесс, то 
		\[
			\{\tau_i \leq n \} \in \filtr_n, ~\forall n \in \N
		\]
		Теперь проверим свойство <<являться Марковским моментом>> всех с.в. поочередно:
		\begin{itemize}
			\item Конечная сумма:
			\[
				\left\{\sum_{k=1}^{m} \tau_k \leq n \right\} 
				=
				\bigcup_{n_1 + \ldots + n_m = n} \left(\bigcup_{k=1}^m \left\{\tau_k \leq n_k \right\} \right)
			\]
			Знаем, что $\left\{\tau_k \leq n_k \right\} \in \filtr_{n_k} \subseteq \filtr_n$, тогда вложенное объединение есть подмножество $\filtr_n$ по свойству сигма-алгебр. Но тогда:
			\[
				\bigcup \left\{ P, ~P\in \filtr_n \right\} \in \filtr_n 
			\]
			так как объединение у нас конечное (одномерных разбиений числа конечно). 
			Получаем:
			\[
				\left\{\sum_{k=1}^{m} \tau_k \leq n \right\}  \in \filtr_n	
			\]
			
			\item Конечное произведение:
			\[
				\left\{\prod_{k=1}^{m} \tau_k \leq n \right\} 
				=
				\bigcup_{n_1 \cdot \ldots \cdot n_m = n} \left(\bigcup_{k=1}^m \left\{\tau_k \leq n_k \right\} \right)
			\]
			Знаем, что $\left\{\tau_k \leq n_k \right\} \in \filtr_{n_k} \subseteq \filtr_n$, тогда вложенное объединение есть подмножество $\filtr_n$ по свойству сигма-алгебр. 
			Далее аналогично предыдущему пункту (разбиения натурального числа на натуральные множители не более чем счётно).
			Получаем:
			\[
				\left\{\prod_{k=1}^{m} \tau_k \leq n \right\}  \in \filtr_n	
			\]
			
			\item Точная верхняя грань: супремум не больше значения -- в точности значит то, что любой из элементов не больше этого значения
			\[
				\left\{\sup _{k} \tau_{k} \leq n \right\} 
				=
				\bigcap_{k} \left\{\tau_k \leq n \right\}
			\]	
			Но $\left\{\tau_k \leq n \right\} \in \filtr_k \subset \filtr_n$. Но мы берем не более чем счетное объединение событий, лежащих в сигма-алгебре и понимаем, что оно также там лежит. Тогда $\left\{\sup _{k} \tau_{k} \leq n \right\} \in \filtr_n$.
			
			
			\item Точная нижняя грань:
			\[
				\left\{\inf _{k} \tau_{k} \leq n \right\} 
				=
				\overline{\left\{\inf _{k} \tau_{k} > n \right\}}
				=
				\overline{\bigcap_{k} \left\{\tau_k > n \right\}}
				=
				\bigcup_{k} \overline{\left\{\tau_k > n \right\}}
				=
				\bigcup_{k} \left\{\tau_k \leq n \right\}
			\]
			Счётное объединение также лежит в сигма-алгебре. Бинго!
		\end{itemize}
		
	\end{proof}
	
	
	
	
	
	
	
	
	
	
	
	
	
	
	
	\newpage
	

	\begin{task2}
		Пусть $(S_n , n \in \N)$ -- простейшее симметричное случайное блуждание. Докажите, что процесс $X_n = S_n^2 - n$ является мартингалом относительно естественной фильтрации процесса $S_n$.
	\end{task2}
	\begin{proof}[Решение]
		\
		Проверим все свойства:
		\begin{itemize}
			\item Согласованность:
			Так как $X_n$ есть борелевская функция относительно $\{S_n\}$, которые порождаются $\Filtr$, то $X_n$ согласован с естественной фильтрацией.
			
			\item Конечность: 
			\[
				\E {|X_n|} = 
				\sum_{k=0}^{n} |k^2 - n| \Pr{S_n^2 - n = k^2 - n} \left(\text{у нас множество значений есть все числа вида $k^2-n$}\right)
				=
			\]
			\[
				= \sum_{k=0}^{n} |k^2 - n| \Pr{S_n = \pm k}
				=
				\sum_{k=0}^{n} |k^2 - n| \left(\Pr{S_n = k} + \Pr{S_n = -k}\right)
			\]
			Можем расписать вероятности, но понимаем, что эта сумма ограничена для любого $n$ (ограничиваем её $n \cdot n^2 = n^3$).
		
			\item Непредсказуемость:
			\[
				\E {X_n | S_{n-1}, \ldots, S_1}
				=
				\E {S_n^2 - n | S_{n-1}, \ldots, S_1}
				=
				\E {\left(S_{n-1} + \xi_n\right)^2 - n | S_{n-1}, \ldots, S_1}
				=
			\]
			\[
				=
				\E {S_{n-1}^2 + \xi_n^2 + 2S_{n-1}\xi_n - n | S_{n-1}, \ldots, S_1}
				=
				\E{\xi_n^2} - n + S_{n-1}^2 + 2S_{n-1}\E{\xi_n}
				=
			\]
			\[
				=
				S_{n-1}^2 + 2S_{n-1}\E{\xi_n} + \E {\xi_n}^2 - (n-1)
				=
			\]
			\[
				\left( S_{n-1} + \E {\xi_n} \right)^2 - (n-1)
				=
				S_{n-1}^2 - (n-1) = X_{n-1}
			\]
			
		\end{itemize}
	
	\end{proof}
	
	
	\newpage
	
	
	
	
	
	
	
	
	
	
	
	
	\begin{task3}
		Пусть $\xi_1 , \ldots , \xi_n , \ldots$ -- такая последовательность случайных величин, что для любого $n$ существует плотность $f_n(x_1 , \ldots , x_n)$ случайного вектора $(\xi_1 , \ldots , \xi_n)$.
		Пусть $\eta_1 , \ldots , \eta_n , \ldots$ -- другая последовательность случайных величин, причем также для любого $n$ существует плотность $g_n (x_1 \ldots , x_n)$ случайного вектора $(\eta_1 , \ldots , \eta_n)$. Докажите, что процесс
		\[
			X_{n}=\frac{g_{n}\left(\xi_{1}, \ldots, \xi_{n}\right)}{f_{n}\left(\xi_{1}, \ldots, \xi_{n}\right)}
		\]
		является мартингалом относительно фильтрации $(\filtr_n = \sigma(\xi_1 , \ldots , \xi_n), n \in \N)$.
	\end{task3}
	\begin{proof} [Решение]
		\
		\begin{itemize}
			\item Согласованность:
			
			Так как $f_n(\xi_1, \ldots, \xi_n)$, как и $g_n$ -- борелевские функции относително $\xi_1, \ldots \xi_n$, то их частное так же. Получается, что процесс согласован с вышеуказанной фильтрацией.
			
			\item Конечность:
			Так как у нас плотности, которые неотицательны, то матожидание модуля есть само матожидание. Посчитаем его:
			\[
				\E {X_n} = \E {\frac{g_{n}\left(\xi_{1}, \ldots, \xi_{n}\right)}{f_{n}\left(\xi_{1}, \ldots, \xi_{n}\right)}}
				=
				\int_{\R^n} \frac{g_{n}\left(x_{1}, \ldots, x_{n}\right)}{f_{n}\left(x_{1}, \ldots, x_{n}\right)} p_{\xi_1, \ldots, \xi_n}(x_1, \ldots, x_n) \d x_1 \ldots \d x_n
				=
			\]
			\[
				=
				\int_{\R^n} \frac{g_{n}\left(x_{1}, \ldots, x_{n}\right)}{f_{n}\left(x_{1}, \ldots, x_{n}\right)}f_{n}\left(x_{1}, \ldots, x_{n}\right) \d x_1 \ldots \d x_n
				=
				\int_{\R^n} g_{n}\left(x_{1}, \ldots, x_{n}\right)\d x_1 \ldots \d x_n
				=
				1.
			\]
			
			\item Непредсказуемость: 
			
			Посчитаем условную плотность: 
			\[
				p_{\xi_1, \ldots, \xi_n | \xi_1, \ldots, \xi_{n-1}} (x_1, \ldots, x_n | x_1 = a_1, \ldots, x_{n-1} = a_{n-1}) 
				=
				{p_{\xi_1, \ldots, \xi_n}(a_1, \ldots, a_{n-1}, x_n) \over p_{\xi_1, \ldots, \xi_{n-1}}(a_1, \ldots, a_{n - 1})}
				=
				{f_n(a_1, \ldots, a_{n-1}, x_n) \over f_{n-1}(a_1, \ldots, a_{n - 1})}
			\]
			
			\[
				\E{X_n | X_{n-1} = b_{n-1}, \ldots, X_1 = b_1}
				= \{\text{вставим нужные значения с.в.}\} =
				\E{X_n | \xi_{n-1} = a, \ldots, \xi_1 = a_1}
				=
			\]
			\[
				=
				\int_{\R} \frac{g_{n}\left(a_{1}, \ldots, a_{n-1}, x\right)}{f_{n}\left(a_{1}, \ldots, a_{n-1}, x\right)} 	{f_n(a_1, \ldots, a_{n-1}, x) \over f_{n-1}(a_1, \ldots, a_{n - 1})}
				\d x
				=
				\int_{\R}{g_{n}\left(a_{1}, \ldots, a_{n-1}, x\right) \over f_{n-1}(a_1, \ldots, a_{n - 1})} \d x
				=
			\]
			\[
				=
				\{\text{выинтегрируем по последней координате}\}
				= {g_{n-1}\left(a_{1}, \ldots, a_{n-1}\right) \over f_{n-1}(a_1, \ldots, a_{n - 1})}
			\]
			Теперь вспомним, как мы ищем УМО. Вместо значений подставим сами с.в.:
			
			\[
				\E{X_n | X_{n-1}, \ldots, X_1}
				=
				{g_{n-1}\left(\xi_{1}, \ldots, \xi_{n-1}\right) \over f_{n-1}(\xi_1, \ldots, \xi_{n - 1})}
				=
				X_{n-1}.
			\]
			
		\end{itemize}
	\end{proof}
	
	
	
	
	
	
	
	
	
	
	
	
	
	\newpage
	
	
	
	\begin{task6}
		Пусть $(N_t , t > 0)$ -- пуассоновский процесс интенсивности $\lambda$. Найдите предел
		п.н. $N_t / t$ при $t \ra +\infty$.
	\end{task6}
	\begin{proof} [Решение]
		Предположим, что $T$ пробегает целые числа.
		
		
		Знаем, что $N_t = N_t - N_0 \sim \Pois(\lambda(t-0)) = \Pois(\lambda t)$
		Воспользуемся ЗБЧ:
		\[
			{N_t \over t} \sim {\Pois(\lambda t) \over t} = {\xi_1 + \ldots + \xi_t \over t}, ~\xi_i \sim \Pois(\lambda) \pn \E {\xi_1} = \lambda 
			\quad 
			(\text{так как матожидание конечно}).
		\]
		
		Теперь подумаем, что делать с нецелостью. Представим $N_t = N_{\lfloor t \rfloor} + N_{\{t\}}$. Для первого слагаемого выполняется ЗБЧ. Рассмотрим второе:
		\[
			 \lim\limits_{t\rightarrow +\infty} {N_{\{t\}} \over t}	< {\Pois(1) \over t}
		\]
		У меня выходила только сходимость по вероятности к нулю, но через какое-то время я вспомнил, что она эквивалентна сходимости по распределению для дискретных с.в.. Покажем сходимость по вероятности к нулю:
		
		\[
			\xi \sim \Pois(1) \quad \lim\limits_{t\rightarrow +\infty} \Pr{{\xi \over t} > \e} = \Pr{\xi > \e t} = 0
			\quad
			\{\text{так как $\e t \ra +\infty$}\}
		\]
		Следовательно 
		\[
			{N_{\{t\}} \over t} < {\xi(1) \over t} \pn 0
			\Rightarrow
			{N_t \over t} \pn \lambda + 0 = \lambda.
		\]
		
	\end{proof}
	



	

\end{document}