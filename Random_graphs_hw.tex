\documentclass[12pt,a4paper]{extarticle}

\usepackage{cmap}                   
\usepackage{mathtext}               
\usepackage[T1,T2A]{fontenc}        
\usepackage[utf8]{inputenc}         
\usepackage[english, russian]{babel} 

\usepackage[top=0.35in, bottom=0.5in, left=0.3in, right=0.3in]{geometry}
\usepackage{mathtools}              
\mathtoolsset{showmanualtags,mathic,centercolon}
\usepackage{amssymb}                
\usepackage{amsthm}                 
\usepackage{amstext}                
\usepackage{amsfonts}               
\usepackage{icomma}                 
\usepackage{enumitem}              
\usepackage{array}                  
\usepackage{multirow}
\usepackage{setspace}

\usepackage{algorithm}              
\usepackage{algorithmicx}           
\usepackage[noend]{algpseudocode}   
\usepackage{listings}              
\renewcommand{\algorithmicrequire}{\textbf{Input:}}              
\renewcommand{\algorithmicensure}{\textbf{Output:}}              
\floatname{algorithm}{Algorithm}                                 
\renewcommand{\algorithmiccomment}[1]{\hspace*{\fill}\{// #1\}}
\newcommand{\algname}[1]{\textsc{#1}}                          
\usepackage{physics}

\usepackage{euscript}               
\usepackage{mathrsfs}               

%% Графика
\usepackage{graphicx}       
\graphicspath{{images/}}            
\usepackage{tikz}  
\usetikzlibrary{patterns}                 
\usepackage{pgfplots}              
\usepackage{circuitikz}


\usepackage{indentfirst}                    
\usepackage{epigraph}                       
\usepackage{fancybox,fancyhdr}              
\usepackage[colorlinks=true,citecolor=blue]{hyperref} 
\usepackage{titlesec}                       
\usepackage[normalem]{ulem}                 
\usepackage[makeroom]{cancel}               
\usepackage{dsfont}

\usepackage{diagbox}
\usepackage{makecell}

\usepackage{csquotes}

\mathtoolsset{showonlyrefs=true}        
\renewcommand{\headrulewidth}{1.8pt}    
\renewcommand{\footrulewidth}{0.0pt}    

\usepackage{forest} 

\usetikzlibrary{arrows,calc}
\usetikzlibrary{quotes,angles}

\usetikzlibrary{positioning,intersections}

\usetikzlibrary{through}

\usepackage{enumitem}

\newenvironment{turing}[2]
{\begin{enumerate}[leftmargin=0pt,labelsep=0pt,align=left,parsep=0pt]
		\item[$#1={}$]``\ignorespaces#2
%		\begin{enumerate}[
			nosep,
			align=left,
			labelwidth=1.5em,
			label=\bfseries\arabic{*}.,
			ref=\arabic{*}
			]}
		{\unskip''\end{enumerate}\end{enumerate}}

\newcommand{\bitem}{\item\hspace*{1em}\ignorespaces}

\usepackage{graphicx}

\newtheorem{definition}{Definition}[section]

\newtheorem*{task}{Task}
\newtheorem*{task0}{Task 0}
\newtheorem*{task1}{Task 1}
\newtheorem*{task2}{Task 2}
\newtheorem*{task3}{Task 3}
\newtheorem*{task4}{Task 4}
\newtheorem*{task5}{Task 5}
\newtheorem*{task6}{Task 6}
\newtheorem*{task7}{Task 7}
\newtheorem*{task8}{Task 8}
\newtheorem*{task9}{Task 9}
\newtheorem*{task10}{Task 10}
\newtheorem*{task11}{Task 11}
\newtheorem*{task12}{Task 12}

\newtheorem{theorem}{Theorem}
\newtheorem{proposal}{Proposal}
\newtheorem{notice}{Notice}
\newtheorem{statement}{Statement}
\newtheorem{corollary}{Corollary}
\newtheorem{lemma}{Lemma}
\newtheorem{observation}{Observation}
\newtheorem{problem}{Problem}
\newtheorem{claim}{Claim}


\newcommand{\note}{\underline{Note:} }
\newcommand{\fact}{\underline{\textbf{Fact}:} }
\newcommand{\example}{\underline{Example:} }


\renewcommand{\Re}{\mathrm{Re\:}}
\renewcommand{\Im}{\mathrm{Im\:}}
\newcommand{\Arg}{\mathrm{Arg\:}}
\renewcommand{\arg}{\mathrm{arg\:}}
\newcommand{\Mat}{\mathrm{Mat}}
\newcommand{\id}{\mathrm{id}}
\newcommand{\aut}{\mathrm{aut}}
\newcommand{\isom}{\xrightarrow{\sim}} 
\newcommand{\leftisom}{\xleftarrow{\sim}}
\newcommand{\Hom}{\mathrm{Hom}}
\newcommand{\Ker}{\mathrm{Ker}\:}
\newcommand{\rk}{\mathrm{rk}\:}
\newcommand{\diag}{\mathrm{diag}}
\newcommand{\ort}{\mathrm{ort}}
\newcommand{\pr}{\mathrm{pr}}
\newcommand{\vol}{\mathrm{vol\:}}
\renewcommand{\mod}{\mathrm{\: mod\:}}
\DeclareMathOperator*\lowlim{\underline{lim}}
\DeclareMathOperator*\uplim{\overline{lim}}
\newcommand{\nd}{\mathbin{\&}}

\newcommand{\X}{\mathbb{X}}
%\newcommand{\D}{\mathbb{D}}
\newcommand{\Y}{\mathbb{Y}}
%\newcommand{\I}{\mathbb{I}}
\makeatletter
\DeclareRobustCommand{\I}{\operatorname{\mathds{I}}\@ifstar\@firstofone\@I}
\newcommand{\@I}[1]{\left\{#1\right\}}
\makeatother

\newcommand{\Z}{\mathbb{Z}}
\newcommand{\Qq}{\mathcal{Q}}
\newcommand{\N}{\mathbb{N}}
%\newcommand{\E}{\mathbb{E}} %
\makeatletter
\DeclareRobustCommand{\E}{\operatorname{\mathds{E}}\@ifstar\@firstofone\@E}
\newcommand{\@E}[1]{\left[#1\right]}
\makeatother

\makeatletter
\DeclareRobustCommand{\D}{\operatorname{\mathbb{D}}\@ifstar\@firstofone\@D}
\newcommand{\@D}[1]{\left[#1\right]}
\makeatother

\makeatletter
\DeclareRobustCommand{\Pr}{\operatorname{\mathds{P}}\@ifstar\@firstofone\@Pr}
\newcommand{\@Pr}[1]{\left[#1\right]}
\makeatother

\renewcommand{\S}{\mathbb{S}}
\newcommand{\Q}{\mathbb{Q}}
\newcommand{\R}{\mathbb{R}} 
\newcommand{\B}{\mathbb{B}}
\renewcommand{\C}{\mathbb{C}}
\renewcommand{\L}{\mathscr{L}}
%\renewcommand{\P}{\mathds{P}}


\newcommand{\orthog}{\mathop{\bot}}
\renewcommand*\d{\mathop{}\!\mathrm{d}}
\renewcommand*\dd{\mathop{}\!\partial}

%\renewcommand{\Pr}{\mathds{P}}
\newcommand{\pn}{\xrightarrow{\text{a. s.}}}
\newcommand{\pp}{\xrightarrow{\mathds{P}}}
\newcommand{\pd}{\xrightarrow{d}}
\newcommand{\ra}{\rightarrow}


\newcommand{\fe}{\varphi}
\newcommand{\e}{\varepsilon}
\newcommand{\ind}{\mathbin{\perp\!\!\!\perp}}
\newcommand{\Gauss}{\mathrm{Gauss}}
\newcommand{\hence}{\longrightarrow}
\newcommand{\bto}{\Longrightarrow}
\newcommand{\Bin}{\mathrm{Bin}}
\newcommand{\Bern}{\mathrm{Bern}}
\newcommand{\Geom}{\mathrm{Geom}}
\newcommand{\Uni}{\mathrm{U}}
\newcommand{\Exp}{\mathrm{Exp}}
\newcommand{\Ko}{\mathrm{Ko}}
\newcommand{\No}{\mathcal{N}}
\newcommand{\Pois}{\mathrm{Pois}}
     
\title{\Huge{ДЗ №3, Случайные графы}}
\author{Павел Захаров}
\date{}
     
     
\begin{document}
	\maketitle

	
	\vspace{\baselineskip}

	
	
	\vspace{\baselineskip}
	\begin{task1}
		Пусть $p \sim c \cdot n^{-k/(k-1)}$, $c > 0$ -- фиксированная константа, а $X_k$ -- это число деревьев фиксированного размера $k$ в $G(n, p)$. Докажите, что $X_k \pd \Pois(\lambda)$ при $n \rightarrow \infty$, где $\lambda = {c^{k-1}k^{k-2} \over k!}$.
	\end{task1}
	
	\begin{proof}[Решение]
		\
		Вспомним теорему Боллобаша:
		\begin{theorem} [Б. Боллобаш]
			Пусть G — строго сбалансированный граф и $np^{m(G)} \ra c > 0$, тогда число копий распределено по пуассоновскому закону $\Pois(\lambda)$, где $\lambda = {c^{v(G)} \over \aut(G)}$.
		\end{theorem}
		Спроцеровав её на нашу задачу и вспомнив, что дерево это строго сбалансированный граф с $m(G) = {k - 1 \over k}$, получим:
		\[
			np^{(k-1)/k} \ra c~ \Rightarrow X_k \pd \Pois(\lambda), \lambda = {c^{k} \over \aut(G)}, \text{где $G$ -- произвольное дерево размера $k$.}
		\]
		Тогда $p \sim c^{k / (k-1)} \cdot n^{-k/(k-1)}$. Вернемся к условию задачи, подставив туда такое $c$:
		\[
			p \sim c^{k / (k-1)} \cdot n^{-k/(k-1)}, ~~ X_k \pd \Pois(\lambda), ~ \lambda = {c^k k^{k-2} \over k!}
		\]
		Но данная $X_k$ учитывает вхождения всех возможных деревьев. Тогда просуммировав по всевозможным деревьям размера $k$ мы должны получить требуемое. 
		
		Введем $tree(k)$ -- множество всех деревьев на $k$ вершинах, различных с точностью до изоморфизма (то есть никакие два дерева из $tree(k)$ не изоморфны).
		
		Тогда осталось доказать:
		\[
			\sum_{g \in tree(k)} {c^k \over \aut(G)} = {c^k k^{k-2} \over k!} \Leftrightarrow 
			\sum_{g \in tree(k)} {k! \over \aut(G)} = k^{k-2}
		\]
		\[
			\text{(здесь мы пользуемся аддиттивостью пуассоновской с.в. по параметру $\lambda$)}
		\]
		Теперь заметим, что выражение справа есть не что иное как число деревьев с данным количеством пронумерованных вершин.
		
		Тогда останется сделать несложное наблюдение: рассмотрим произвольный граф из $tree(k)$ и сосчитаем число его возможных вхождений  на пронумерованных вершинах. 
		
		Так как все вершины пронумерованы, то и порядков их взаимного расположения будет $k!$, в точности как число перестановок длины $k$. Но это не будет в точности количество возможных вхождений данного дерева, так как мы имеем повторы.
		
		Давайте их сосчитаем. Предположим у нас есть две нумерации вершин, при которых деревья совпадают друг с другом. Но ведь тогда это в точности два автоморфизма этого дерева. Тогда понятно, что таких совпадающих деревьев у нас будет $\aut(g)$ для каждого дерева. Тогда вхождений каждого дерева из $tree(k)$ будет ${k! \over \aut(g)}$. Тогда сумма по всем различным деревьям будет равна общему количеству деревьев на $k$ пронумерованных вершинах, что мы и хотели доказать.
		
	\end{proof}












	\newpage
	
	
	
	
	
	
	
	\begin{task2}
		\
		\begin{itemize}
			\item[а)] Пусть $G = C_3 \sqcup C_3$ -- два непересекающихся треугольника, $pn \ra c > 0$. Докажите, что тогда $X_G \pd {1\over 2} Z(Z - 1)$ при $n \ra \infty$, где $Z \sim \Pois(c^3/6)$.
			
			\item[b)] Пусть $G = C_3 \sqcup C_4$ -- два непересекающихся цикла длины 3 и 4, $pn \ra c > 0$. Докажите, что тогда $X_G \pd Z_1Z_2$ где $Z_1$ и $Z_2$ независимы,  $Z_1 \sim \Pois(c^3/6)$, $Z_2 \sim \Pois(c^4/8)$.
		\end{itemize}
	\end{task2}
	\begin{proof}[Решение]
		\
		\begin{itemize}
			\item[a)] Заметим, что если все треугольники не пересекаются, то ответ такой, какйо нам надо (так как треугольники распределеные по Пуассону с $\lambda = c^3 / 6$, то мы берем один, затем берем один из оставшихся и перемножаем по независимости и делим на два, так как порядок взятия неважен). 
			
			Покажем, что с вероятностью один все треугольники не пересекаются. Будем в качестве пересечения брать пересечение по ребру, так как если пересечение только по вершине, то треугольники все равно берутся независимо. 
			
			Рассмотрим фигуру -- пара треугольников с общим ребром. Это единственный способ пересечения двух треугольников, такой, что они не будут совпадать. Посчитаем вероятность появления такой фигуры в нашем случайном графе.
			
			Мы выбираем 4 вершины и проводим 5 ребер получаем вероятность:
			\[
				\binom{n}{4} p^5 = {n(n-1)(n-2)(n-3) \over n^5}\cdot c^5 \Rightarrow 0
			\]
			Отлично! Еще раз напомню, что так как в пределе все треугольники не пересекаются, то просто перемножаем наши случайные величины. 
			
			\item[b)] То же самое: 
			$Z_1$ -- распределение $C_3$, $Z_2$ -- $C_4$. Тогда мы хотим точно так же показать независимость. У нас два варианта пересечения $C_3$ и $C_4$: по одному ребру или по двум.
			\begin{itemize}
				\item[По одному:] Пять вершин и шесть ребер. Имеем вероятность, что такая штука будет в графе:
				\[
					\binom{n}{5} p^6 = {\binom{n}{5} \over n^6}\cdot c^6 \ra 0.
				\]
				\item[По двум:] Четыре вершин и пять ребер. Имеем вероятность, что такая штука будет в графе:
				\[
					\binom{n}{4} p^5 = {\binom{n}{4} \over n^5}\cdot c^5 \ra 0.
				\]
			\end{itemize}
			Тогда вероятность пересечения $C_3$ и $C_4$ стремится к нулю. Тогда мы можем выбирать их независимо, то есть перемножать с.в., отвечающие их распределениям. 
		\end{itemize}
	\end{proof}
	
	
	
	
	
	
	\newpage
	
	
	
	
	
	
	
	\begin{task5}
		\
		Пусть $np \ra c, c > 1$. Обозначим через $X_n$ -- число ребер в гигантской компоненте $G(n, p)$. Найдите такую функцию $\theta = \theta(c) > 0$, что
		\[
			{X_n \over n} \pp \theta \text{ при } n \ra +\infty
		\]
	\end{task5}
	\begin{proof}[Решение]
		\
		Будем делать всё просто. Имеем компопенту размера $\beta n$. Тогда выкинем из неё остов и будем считать количество остальных рёбер. У нас мест под рёбра -- $\binom{\beta n}{2} - \beta n + 1 = {\beta n(\beta n-1) \over 2} + 1$.
		Вероятность взять каждое из них ${c\over n}$. Тогда по ЗБЧ имеем в пределе:
		\[
			{X_n \over n} \pp \lim\limits_{n\rightarrow +\infty} c{{\beta n(\beta n-1) \over 2} + 1 \over n^2} = {c\beta^2 \over 2}
		\]
		Здесь $\beta \in (0; 1)$ -- корень уравнения $\beta + e^{-\beta c} = 1$.
	\end{proof}
	
	
	
	
	
	
	
	
	\newpage
	
	
	
	
	
	
	
	
	\begin{task6}
		\
		Пусть $np \ra c, ~c > 1$. Докажите, что с вероятностью, стремящейся к 1, гигантская компонента является единственной сложной компонентой в случайном
		графе $G(n, p)$.
	\end{task6}
	\begin{proof}[Решение]
		\
		Перед тем как переходить к доказательсву заметим, что в теореме о гигантской компоненте нам достаточно сходимости $np \ra c, ~ c > 1$.
		
		На лекции мы доказывали, что конфигураций сложных компонент на $k$ вершинах не больше чем $k^2k!$, то есть вероятность получить хотя бы одну не превосходит $k^2k!p^{k+1} < k^2k!p^{k}$. 
		
		Тогда рассмотрим вероятность того, что негигантская компонента является сложной. Из теоремы о гигантской компоненты мы знаем, что размеры остальных компонент не превосходят ${16c \over (1-c)^2} \ln n$. Подставим это значение в вышенаписанную формулу:
		\[
			k^2k!p^k = \left\{k = {16c \over (1-c)^2} \ln n =: d\ln n = \ln n^d  \right\} = \ln^2 n^d(\ln n^d)!\left({c \over n}\right)^{\ln^d n}
		\]
		Мы хотим показать, что вероятность того, что негигантская компонента сложная стремится к нулю. Тогда оценим сверху:
		\[
			\ln^2 n^d(\ln n^d)!\left({c \over n}\right)^{\ln^d n} 
			\sim \{\text{формула Стирлинга}\} \sim
			\ln^2 n^d\cdot \ln n^d \cdot n^d\left({c \over n}\right)^{d\ln n} 
			\leq 
		\]
		\[
			\leq {n^{4d} c^{d\ln n} \over n^{d\ln n}} 
			\sim 
			{c^{d\ln n} \over n^{d\ln n}} = \left({c \over n}\right)^{d\ln n}
		\]
		
		Тогда вероятность того, что негигантская компонента несложная оценивается как 
		\[
			1 - \left({c \over n}\right)^{d\ln n}
		\]
		Хотим что-то сказать про вероятность того, что все компоненты несложные. Очень хочется оценить это просто возведением нашего результата в степень равную числу компонент. Но не можем, так как события не независимы. Стоп. Но ведь все компоненты являются обособленными друг от друга, и ребра появляются в разных частях графа независимо друг от друга. Тогда о никакой зависимости не может идти и речи. Тогда распишем по независимости:
		\[
			\Pr{\text{все малые компоненты не являются сложными}} \geq \left(1 - \left({c \over n}\right)^{d\ln n}\right)^{\text{число компонент}} \geq \left(1 - \left({c \over n}\right)^{d\ln n}\right)^{n}
		\]
		А тут есть школьный трюк -- неравенство Бернулли. Так как $n \ra +\infty$, а $- \left({c \over n}\right)^{d\ln n} > -1$, то пользование им корректно.
		\[
			\left(1 - \left({c \over n}\right)^{d\ln n}\right)^{n} \geq 1 - n \left({c \over n}\right)^{d\ln n}
		\]
		Ну и напоследок скажем, что это выражение стремится к единице, так как вычитаемое стремится к нулю с ростом $n$. Тогда все малые компоненты являются не-сложными. 
		
		\vspace{\baselineskip}
		
		
		
		
		Теперь покажем, что гигантская компонента является сложной. 
		Сделаем вид, что мы знаем точное её местоположение (запустим весь процесс построения, а по окончании его запустим его ещё раз, с теми же исходами, (что-то, похожее на то, как в статистике -- имеем лишь  исходы, но у нас есть информация о распределении)). 
		
		Отметим первый момент времени, в котором гигантская компонента стала деревом.
		Теперь покажем, что помимо уже добавленных рёбер мы обязательно добавим ещё хотя бы два. Оставим без доказательства тот простой факт, гласящий то, что если в дерево добавить два ребра, то в нём будет два цикла. 
		
		Посчитаем вероятность того, что мы проведем ноль ребер, обозначив размер гигантской компоненты за $bn$. Так как мы их проводим независимо, то это будет просто вероятность не провести ребро в степени количество свободных мест под рёбра, то есть:
		\[
			\left(1 - {c\over n}\right)^{\binom{bn}{2} - (bn-1)} \sim \left(1 - {c\over n}\right)^{n^2} \sim e^{-cn/2} \rightarrow 0
		\]
	 	Отлично. Теперь по аналогии посчитаем вероятность провести ровно одно ребро. Для этого возьмём количество способов провести его, а дальше всё точно так же:
		\[
			\left(\binom{bn}{2} - (bn - 1)\right){c \over n}\left(1 - {c\over n}\right)^{\binom{bn}{2} - (bn-1) - 1} \sim n^2{c\over n}\left(1 - {c\over n}\right)^{n^2} \sim ne^{-bcn/2} \rightarrow 0
		\]
		Тогда мы показали, что вероятность добавить одно или ноль ребер стремится к нулю, то есть с вероятностью один мы добавим хотя бы два ребра, то есть гигантская компонента будет сложная.
	
	\end{proof}	
	
	
	
	\newpage
	
	
	
	
	
	
	
	\begin{task7}
		\
		Пусть $X_n$ -- число циклов в случайном графе $G(n, 1/n)$. Докажите, что
		$\E{X_{n}} \sim {1\over4} \ln n$.
	\end{task7}
	\begin{proof}[Решение]
		\
		Знаем из лекций, что матожидание числа циклов в случайном графе $G(n, p)$ записывается в следущем виде:
		\[
			\E {X_n} = \sum_{k=3}^{n} \binom{n}{k} {k!\over 2k}p^k = \sum_{k=3}^{n} {n! \over (n-k)!n^{k}}\cdot {1\over 2k}
		\]
		Разделим эту сумму на три части:
		\[
			\sum_{k=3}^{n} {n! \over (n-k)!n^{k}}\cdot {1\over 2k}
			= \sum_{k=3}^{\sqrt{n / \ln n}} {n! \over (n-k)!n^{k}}\cdot {1\over 2k}
			+
			\sum_{k=\sqrt{n / \ln n}}^{\sqrt{n \ln n}} {n! \over (n-k)!n^{k}}\cdot {1\over 2k}
			+
			\sum_{k=\sqrt{n \ln n}}^{n} {n! \over (n-k)!n^{k}}\cdot {1\over 2k}
		\]
		Оценим каждую из частей:
		\begin{itemize}
			\item[1)] Покажем сходимость к тому, к чему надобно:
			\[
				\sum_{k=3}^{\sqrt{n / \ln n}} {n! \over (n-k)!n^{k}}\cdot {1\over 2k} \sim \{ \text{так как $n - k \sim n$, то дваждый воспользуемся формулой Стирлинга}\}	
				\sim
			\]
			\[
				\sim 	\sum_{k=3}^{\sqrt{n / \ln n}} {\sqrt{2\pi n} \left(n \over e\right)^n \over \sqrt{2\pi (n-k)} \left(n-k \over e\right)^{n-k}n^{k}}\cdot {1\over 2k}
				\sim 
				\sum_{k=3}^{\sqrt{n / \ln n}} \sqrt{n \over n - k} \left(n \over n - k\right)^{n-k}  e^{-k}\cdot {1\over 2k} 
				\sim
			\]
			\[
				\sim \{\text{предельный переход в формуле экспоненты}\} 
				\sim 
				\sum_{k=3}^{\sqrt{n / \ln n}} \sqrt{n \over n - k}{1\over 2k} 
			\]
			Теперь скажем пару слов про первый множитель. С одной стороны мы можем ограничить его снизу единицей. С другой -- взять его максимально возможным, то есть при $ k = \sqrt{n / \ln n}$. Но в таком случае с ростом $n$ выражение стремится к единице, то есть начиная с определенного момента может лежать в $\varepsilon$-окрестности единицы для любого $\varepsilon$. Чтож, мы и сверху ограничили его единицей. Тогда можем смело отбросить его и посмотреть на то, что осталось.
			
			Теперь воспользуемся школьными знаниями:
			\[
				\sum_{k = 0}^{n} {1\over k}\sim \ln n
			\]
			Тогда 
			\[
				\sum_{k=3}^{\sqrt{n / \ln n}} \sqrt{n \over n - k}{1\over 2k} \
				\sim
				\sum_{k=3}^{\sqrt{n / \ln n}} {1\over 2k} 
				\sim {1\over 2}\ln {\sqrt{n / \ln n}} = {1\over 4} \ln n - {1\over 4} \ln \ln n \sim {1\over4}\ln n;
			\]
			\item[2)] Покажем малость:
			\[
				\sum_{k=\sqrt{n / \ln n}}^{\sqrt{n \ln n}} {n! \over (n-k)!n^{k}}\cdot {1\over 2k}
			\]
			Все оценки совершенно аналогичные. Оценивая гармонический ряд получаем:
			
			\[
				\ln\left( \sqrt{n \ln n} \right) - \ln\left( \sqrt{n /\ln n} \right) = \ln \ln n = o(\ln n);
			\]
			\item[3)] Покажем малость:
			\[
				\sum_{k=\sqrt{n \ln n}}^{n} {n! \over (n-k)!n^{k}}\cdot {1\over 2k}
			\]
			Топорно ограничить с помощью формулы Стирлинга не получается. А думать не хочется, поэтому возьмём другие оценки. 
			
			Мы хотим показать сходимость этого добра к нулю, поэтому ограничим сверху.
			
			Преобразуем первый из множителей:
			\[
				{n! \over (n-k)!n^{k}} = {n \over n}  {n - 1\over n}  \cdots  {n-k \over n} = \exp{\ln{n \over n} + \ln{n - 1\over n} + \cdots + \ln{n-k \over n}} = 
			\]
			\[
				=  \exp{-\left(\ln{n \over n} + \ln{n\over n - 1} + \cdots + \ln{n \over n - k}\right)} = \{\ln (1 + x) \sim x, ~x\rightarrow 0\} =
			\]
			\[
				=\exp{-\left({0 \over n} + {1\over n - 1} + \ldots + {k \over n - k}\right)} \leq
			\]
			\[
				\leq \exp{-\left({0 \over n} + {1\over n} + \ldots + {k \over n}\right)} = \exp{-{k(k+1) \over n}} \sim \exp{-{k^2 \over n}}
			\]
			
			Заметим, что это выражение будет наибольшим при $k = \sqrt{n \ln n}$. Тогда в нашей сумме заменим все первые множители на данное выражение, с подставленным значением $k$:
			\[
				\sum_{k=\sqrt{n \ln n}}^{n} {n! \over (n-k)!n^{k}}\cdot {1\over 2k} \leq \sum_{k=\sqrt{n \ln n}}^{n} \exp{-{n \ln n \over n}} {1\over 2k} 
				=
				{1\over 2n} \sum_{k=\sqrt{n \ln n}}^{n} {1\over k} \sim {\ln n \over n} = o(\ln n).
			\] 
			Малость показана.
		\end{itemize}
		Теперь мы просто складываем все три суммы воедино и получаем нечто, ведущее себя ассимптотически как первое слагаемое, то есть как ${1\over 4} \ln n$.
		
	\end{proof}
	
	

	
	

\end{document}